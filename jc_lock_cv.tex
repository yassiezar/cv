\documentclass[11pt,a4paper]{awesome-cv}        % possible options include font size ('10pt', '11pt' and '12pt'), paper size ('a4paper', 'letterpaper', 'a5paper', 'legalpaper', 'executivepaper' and 'landscape') and font family ('sans' and 'roman')

\fontdir[fonts/]

\colorlet{awesome}{awesome-red}

\definecolor{text}{HTML}{000000}
\definecolor{darktext}{HTML}{000000}
\definecolor{graytext}{HTML}{000000}
\definecolor{lighttext}{HTML}{000000}

\setbool{acvSectionColorHighlight}{true}

% character encoding
%\usepackage[utf8]{inputenc}                       % if you are not using xelatex ou lualatex, replace by the encoding you are using

% adjust the page margins
%\usepackage[top=1.5cm, bottom=1.5cm, left=1cm, right=2cm, scale=0.8]{geometry}
\geometry{left=1.4cm, top=.8cm, right=1.4cm, bottom=1.8cm, footskip=.5cm}

% personal data
\photo[circle,edge,left]{pp.png}
\name{Jacobus Lock,}{PhD}
\position{Robotics Engineer {\enskip\cdotp\enskip} Team Leader}
%\position{Recent PhD graduate and researcher seeking a transition from academia to professional R\&D opportunities in robotics engineering}
\address{Farnham, UK}% optional, remove / comment the line if not wanted; the "postcode city" and and "country" arguments can be omitted or provided empty
\mobile{+44 (0)7449 739341}                   % optional, remove / comment the line if not wanted
\email{jaycee.lock@gmail.com}                               % optional, remove / comment the line if not wanted
\homepage{jclock.co.uk}                         % optional, remove / comment the line if not wanted
\linkedin{jacobus-lock}
\github{jayceelock}                         % optional, remove / comment the line if not wanted
\googlescholar{MLYdvnwAAAAJ}{JC Lock}

\begin{document}
\makecvheader

\makecvfooter{\today}{Jacobus C. Lock~~~·~~~Cirriculum Vitae}{\thepage}

\cvsection{Work Experience}

\begin{cventries}
  \cventry
  {Senior Software Engineer}
  {Mathworks}
  {Cambridge, UK}
  {April 2023 -- Present}
  {
    \begin{cvitems}
      \item Led development of new feature to improve cross-product compatibility between AUTOSAR and Simulink Real-Time models
      \item Conducted regular meetings to gather customer and technical feedback to refine feature requirements
    \end{cvitems}
  }
  \cventry
  {Senior Robotics Software Engineer, Team Leader}
  {Fox Robotics}
  {Farnham, UK}
  {June 2021 -- April 2023}
  {
    \begin{cvitems}
      \item Implemented a new, modern development and project management practises to improve development efficiency, standardise contributions and improve documentation
      \item Led a team through a code redesign and rebuilding phase to improve the existing codebase using modern C++ and Python design principles
      \item Mentored junior engineers during the redesign and rebuilding process
      \item Built IoT devices and associated firmware to enable remote human-robot-webserver interactions
      \item Contributed to the high-level system design of an agriculture robot and ancillary hardware to work alongside and interact with humans
      \item Successfully delivered live, spectator-controlled demonstrations for prospective clients within budget and on time
    \end{cvitems}
  }
  \cventry
  {Post-Doctorate Research Fellow}
  {University of Lincoln}
  {Lincoln, UK}
  {April 2020 -- June 2021}
  {
    \begin{cvitems}
      \item Implemented a ROS-based sensor fusion and vehicle localisation system for a sprayer vehicle
      \item Designed and built surface modelling software to estimate the shape of crops and terrain 40m in front of a vehicle to within 5cm using LiDAR
      \item Designed a controller that uses localisation and surface data to keep sprayer booms 30cm from the surface to avoid hitting crops
    \end{cvitems}
  }
  \cventry
  {Robotics Engineer (contract)}
  {SAGA Robotics Ltd.}
  {Lincoln, UK}
  {Jan. 2020 -- April 2020}
  {
    \begin{cvitems}
      \item Coordinated with a colleague to translate their robot arm path planning software into a ROS-compatible simulation plugin 
      \item Built a Gazebo simulation environment to test the arm on different platforms via MoveIt
      \item Achieved stated objectives 2 months ahead of schedule
    \end{cvitems}
  }
  \cventry
  {Research Assistant}
  {University of Lincoln}
  {Lincoln, UK}
  {April 2019 -- Jan. 2020}
  {
    \begin{cvitems}
      \item Designed and built a robust solution to mount 2 Panda 7-DoF arms onto Saga Robotics' Thorvald robot platform
      \item Collaborated with project team and built a ROS-compatible robot model of the complete robot system 
      \item Integrated MoveIt path planning into simulation environment
    \end{cvitems}
  }
\end{cventries}

\cvsection{Education}

\begin{cventries}
  \cventry
  {PhD, Computer Science}
  {University of Lincoln}
  {Lincoln, UK}
  {April 2016 -- March 2020}
  {
    \begin{cvitems}
      \item Google-funded project investigating the efficacy of active vision techniques when applied to human tasks to assist visually impaired users
      \item Built and integrated an AI-driven object detector and POMDP policy framework onto an Android app
      \item Achieved an 8-fold time performance improvement over remotely-driven alternatives
      \item Published several academic papers with international collaborators and presented my work at various conferences
      \item App code and thesis document available at \url{https://bit.ly/38TJHPv} and \url{https://bit.ly/3vCo3ZX} respectively
    \end{cvitems}
  }
  \cventry
  {M.Eng by research, Mechatronic Engineering}
  {Stellenbosch University}
  {Stellenbosch, South Africa}
  {Jan. 2014 -- Dec. 2015}
  {
    \begin{cvitems}
      \item Devised a computer vision-based method to easily measure a drone's pose estimation error in flight so that they can more effectively calibrate mirrors in a solar thermal power plant
      \item Methodologies included using a camera and OpenCV determine a drone's pose and train a neural network model to give error bounds
      \item Took several optional modules: Advanced control systems, Advanced dynamics, Advanced numerical methods, Computer vision, Machine learning
      \item Assisted lecturers during tutorial sessions for various undergraduate subjects throughout my masters studies
    \end{cvitems}
  }
  \cventry
  {B.Eng, Mechatronic Engineering}
  {Stellenbosch University}
  {Stellenbosch, South Africa}
  {Jan. 2010 -- Dec. 2013}
  {
    \begin{cvitems}
      \item Volunteered as a student mentor for 2 years to help new students acclimatise to campus life
      \item Nominated for best dissertation presentation in my final year
    \end{cvitems}
  }  
\end{cventries}

\newpage
\cvsection{Skills}

\begin{cvskills}
  \cvskill
  {Programming}%
  {C, C\texttt{++}, Rust, Java, Python, Kotlin, Latex, Git, Agile development}%
  \cvskill
  {Data Analysis}%
  {Matlab, NumPy, Pandas}%
  \cvskill
  {Embedded systems}
  {Arduino, Android, ROS, Embedded electronics}
  \cvskill
  {Language}
  {Afrikaans, English, German (learning), Dutch (learning)}
  \cvskill
  {Web hosting}
  {Bare metal VPS, Django, REST API, Docker}
  \cvskill
  {Soft skills}
  {Technical writing, technical and customer presentation}
\end{cvskills}

\cvsection{Extracurricular Activity}

\begin{cventries}

  \cventry
  {Personal}
  {Personal Tinkering Projects}
  {Lincoln, UK}
  {2020 -- Present}
  {
    \begin{cvitems}
      \item Built Django-based site to host my personal web page and back up my personal data, currently migrating the server to a home-based machine
      \item Designing an Arduino-controlled meat drying box to make biltong
      \item Built a home-based media server
      \item Involved with open-source SMACC library development
    \end{cvitems}
  }
  \cventry
  {Upwork.com}
  {Freelance Software Engineer}
  {Lincoln, UK}
  {2017 -- 2019}
  {
    \begin{cvitems}
      \item Made an Android app that geo-references locations where sounds of a specific frequency are detected using an FFT
      \item Built a Raspberry Pi-controlled model train town than can be controlled globally from a web interface I built
      \item Designed and implemented a vending machine that allows cashless purchases to be made using QR codes, NFC or student ID authentication 
      \item Made an app-controlled lock management system where a user can set the status of a group of locks from their mobile
    \end{cvitems}
  }
  \cventry
  {Academic Journals}
  {Academic Paper Reviewer}
  {Lincoln, UK}
  {2020 -- Present}
  {
    \begin{cvitems}
      \item Reviewed multiple paper submissions to technology-focussed journals
    \end{cvitems}
  }
\end{cventries}

\cvsection{Hobbies and Interests}

\begin{cventries}

  \cventry
  {
  }
  {}
  {}
  {}
  {
    \begin{cvitems}
      \item Learning about, travelling to and exploring historic areas
      \item Studying basic economics, particularly those of developing regions
      \item Outdoor exercise and running (completed 3 half-marathons)
      \item Boxing, touch rugby
    \end{cvitems}
  }
\end{cventries}

\end{document}
